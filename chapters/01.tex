
\chapter{Introduction}
\label{chap:introduction}

In recent years, mobile gaming has seen rapid growth, yet many games fail to deliver the complexity and engagement necessary for long-term player retention\footnote{Dice games have a rich history dating back thousands of years, with early examples found in ancient cultures such as Mesopotamia and Egypt. The simplicity and randomness of dice rolls have made them a staple in games of chance and strategy.}. Traditional dice games, with their simple rules and strategic depth, offer an opportunity to innovate within the mobile gaming space. However, many mobile dice games often fall short due to \emph{predictable AI behavior} and a lack of meaningful player feedback ~\cite{bib:appannie}. This can lead to gameplay that feels repetitive, with easily exploitable strategies and minimal opportunities for players to improve ~\cite{bib:yannakakis}. Such shortcomings often result in disengaged players who struggle with steep learning curves, lacking any real insights into their progress.

\section{Objectives}
This thesis presents the development of a mobile application designed to support and enhance dice games through the integration of \emph{artificial intelligence (AI)} and \emph{image recognition technology}. The goal of this project is to create an immersive, intelligent, and interactive mobile gaming experience by addressing the following key objectives:
\begin{enumerate}
    \item To implement an adaptive AI system that adjusts to the player’s skill level across various dice games.
    \item To develop an image recognition system that allows users to simulate dice rolls using their device's camera.
    \item To enhance the user experience by providing dynamic gameplay, real-time interaction, and a seamless transition between the physical and digital worlds.
\end{enumerate}

\section{Project Requirements}
This section outlines the essential requirements that the project aims to fulfil to achieve its objectives.

\begin{itemize}
    \item Support for classic dice games such as \emph{Pig}, \emph{Balut}, and \emph{Greed}.
    \item Integration of an AI engine capable of adapting to player strategies and behaviors.
    \item Real-time image recognition to detect dice patterns and simulate rolls.
    \item A mobile application built with \emph{Kotlin} and \emph{Jetpack Compose} for a responsive and intuitive interface.
    \item Efficient performance with real-time gameplay and minimal latency.
\end{itemize}

The application is built using \emph{Kotlin} and \emph{Jetpack Compose}, offering a blend of traditional dice game mechanics and modern technological enhancements. The AI component provides a \emph{challenging and adaptive opponent}, while the image recognition system offers a seamless gaming experience that bridges the gap between the physical and virtual worlds.

\section{Thesis Structure}
This thesis is structured into seven chapters, each addressing a specific aspect of the project.

\begin{enumerate}
    \item Chapter ~\ref{chap:introduction} introduces the mobile gaming landscape, along with the project's goals and requirements.
    \item Chapter ~\ref{chap:requirements-and-tools} outlines the system requirements, architecture design, data model, user interface, and AI model design.
    \item In Chapter ~\ref{chap:problem-analysis}, the problem is accessed, exploring the mechanics of dice games, the role of artificial intelligence (AI) in gaming, and the architecture of mobile applications.
    The external specifications are detailed in Chapter~\ref{chap:external-specifications}, covering the development of the Android application, integration with RoboFlow API, and the implementation of player analytics and game mechanics.
    \item Chapter ~\ref{chap:internal-specifications} covers the internal specifications, including unit testing, integration testing, and user acceptance testing, as well as performance analysis and evaluation of the AI model.
    \item Verification and validation processes are discussed in Chapter ~\ref{chap:verification-and-validation}, ensuring the system meets its requirements and functions correctly.
    \item Finally, Chapter ~\ref{chap:conclusions} summarizes the key elements of the thesis, reflecting on its achievements, limitations, and offering suggestions for future improvements.
\end{enumerate}
