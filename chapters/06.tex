\chapter{Verification and Validation}
\label{chap:verification-and-validation}

Ensuring the reliability, functionality, and usability of software applications necessitates rigorous verification and validation processes. Verification confirms that the application is built according to design specifications, while validation ensures it fulfils user needs.

This chapter provides an overview of our quality assurance methods during application development. Specifically, it outlines the testing paradigm, test case designs, testing scope, detected and resolved bugs, and experimental results.

\section{Testing}
The Table~\ref{tab:testing_summary} summarizes the different testing types, their purposes, and scopes for this application.

\begin{table}[ht!]
    \centering
    \caption{Summary of Testing Types and Scope}
    \label{tab:testing_summary}
    \begin{tabular}{|p{4cm}|p{4cm}|p{4cm}|}
    \hline
    \textbf{Test Type} & \textbf{Purpose} & \textbf{Scope} \\ \hline
    Unit Testing       & Validate individual components or functions. & Core game logic, utility functions. \\ \hline
    Integration Testing & Ensure correct interaction between modules. & Game logic and UI, and image recognition. \\ 
    \hline
    System Testing     & Verify the complete application works as intended. & End-to-end gameplay and Image recognition. \\ \hline
    Regression Testing & Identify defects after changes to the codebase. & Post-fix validation of all modules. \\ \hline
    Performance Testing & Measure responsiveness and stability under load. & Frame rate, AI performance with multiple players. \\ \hline
    \end{tabular}
\end{table}

The V-Model testing paradigm guided the verification and validation of this application. A structured approach, particularly suitable where high-quality assurance is critical~\cite{bib:vmodel}, the V-Model expands on the traditional Waterfall Model by emphasizing a parallel relationship between development and testing phases (Figure~\ref{fig:vmodel_diagram}). 

\begin{figure}[ht!]
    \centering
    \begin{tikzpicture}[
        box/.style={draw, rectangle, fill=gray!20, text width=3cm, align=center, minimum height=1cm}, 
        >=Stealth
    ]
        % Nodes
        \node[box] (requirements) {Requirements Analysis};
        \node[box, right=1.5cm of requirements] (acceptance) {Acceptance Testing};
        \node[box, below=1.5cm of requirements] (design) {System Design};
        \node[box, below=1.5cm of acceptance] (integration) {Integration Testing};
        \node[box, below=1.5cm of design] (component) {Component Design};
        \node[box, below=1.5cm of integration] (system) {System Testing};
        \node[box, below=1.5cm of component] (coding) {Coding};
        \node[box, below=1.5cm of system] (unit) {Unit Testing};
        
        % Arrows
        \draw[->] (requirements) -- (design);
        \draw[->] (design) -- (component);
        \draw[->] (component) -- (coding);
        \draw[->] (coding) -- (unit);
        \draw[->] (unit) -- (system);
        \draw[->] (system) -- (integration);
        \draw[->] (integration) -- (acceptance);
        \draw[->] (acceptance) -- +(-0.2,0) |-  (requirements);
    \end{tikzpicture}
    \caption{Simplified V-Model Diagram}
    \label{fig:vmodel_diagram}
\end{figure}

Each development activity was paired with a corresponding testing phase to ensure systematic validation at every stage; for example, requirements analysis was linked to acceptance testing. The V-Model's clarity and emphasis on accountability made it an ideal paradigm for this project, with its aim to deliver a reliable, user-friendly application.

The V-Model incorporates two fundamental types of testing:
\begin{itemize}
    \item \textbf{Verification}: Involves activities like reviews, inspections, and static testing to confirm the application meets specified requirements. For example, unit tests confirm that individual components behave as designed.
    \begin{lstlisting}[language=Kotlin, caption=Unit Test for Game Initialization, label=lst:unit_game_init]
    @Test
    fun `test game initialization`() {
        val gameState = balutGameManager.initializeGame()   
        // Verify initial state
        assertEquals(2, gameState.playerScores.size)
        assertTrue(gameState.playerScores[0]?.isEmpty() == true)
        assertEquals(1, gameState.currentRound)
        assertFalse(gameState.isGameOver)
    }
    \end{lstlisting}
    \item \textbf{Validation}: Consists of dynamic testing where the final product is evaluated to confirm it meets user requirements and needs. For example, integration tests are done to confirm correct interactions between software modules.
    \begin{lstlisting}[language=Kotlin, caption=Integration Test for Dice Detection, label=lst:integration_dice_detection]
    @Test
    fun `detectDice should update state correctly`() = runTest {
        val detections = listOf(mockk<Detection>())
        coEvery { repository.detectDice(any()) } returns detections
        viewModel.detectDice(mockBitmap)
        assert(viewModel.detectionState.value is DetectionState.Processing)
        testDispatcher.scheduler.advanceUntilIdle()
        assert(viewModel.detectionState.value is DetectionState.Success)
    } 
    \end{lstlisting}
\end{itemize}


\section{Test Cases and Testing Scope}

The application's testing strategy utilized full and partial approaches, as outlined below.

\subsection{Full Testing}

Full testing focused on critical components of the application:
\begin{itemize}
    \item \textbf{Game Logic}: Included comprehensive tests of all game rules and scoring mechanics, targeting game manager components:
    \begin{lstlisting}[language=Kotlin, caption=Unit Test for Game Logic, label=lst:game_logic_unit]
    @Test
    fun `test player rolls 1 and loses turn`() {
        val initialState = pigGameManager.initializeGame()
            .copy(currentPlayerIndex = 0)
        val updatedState = pigGameManager.handleTurn(initialState, 1) 
        assertEquals(AI_PLAYER_ID.hashCode(), updatedState.currentPlayerIndex)
        assertEquals(0, updatedState.currentTurnScore)
        assertEquals(0, updatedState.playerScores[0])
    } 
    \end{lstlisting}

    \item \textbf{Integration Testing}: Validated the interaction of various application modules and their integration within a single system.
    \begin{lstlisting}[language=Kotlin, caption=Integration Test for Game View Model, label=lst:integration_game_view_model]
    @Test
    fun `integration test gameViewModel`() = runTest {
        val gameState = viewModel.handleTurn(listOf(1))
       assert(gameState.currentPlayerIndex == AI_PLAYER_ID.hashCode())
    } 
    \end{lstlisting}
    
    \item \textbf{User Interface}: Usability and responsiveness of the UI were assessed across different devices and screen sizes.
\end{itemize}

\subsection{Partial Testing}

Partial testing was applied to less critical components, with an emphasis on functionality isolation:

\begin{itemize}
    \item \textbf{Unit Testing}: Individual methods were assessed using JUnit tests focused on isolated code.
    \begin{lstlisting}[language=Kotlin, caption=Unit Test for Dice Rolling, label=lst:unit_dice_rolling]
    @Test
    fun `test roll dice for Greed game`() = runTest {
        val results = diceManager.rollDiceForBoard(GameBoard.GREED.modeName)
        assertEquals(6, results.size)
        results.forEach { dice ->assertTrue(dice in 1..6)}
    }
    \end{lstlisting}
    \item \textbf{Regression Testing}: Existing functionality was assessed post-fixes and/or new additions to the code base, specifically to check and confirm errors fixed do not manifest and do not effect further components.
    \begin{lstlisting}[language=Kotlin, caption=Regression Test for Game Completion, label=lst:regression_game_completion]
    @Test
    fun `test game completion`() {
        // Create a state where all categories are filled except one
        val allCategories = BalutGameManager.CATEGORIES
            .associateWith { 10 }
            .toMutableMap()
        allCategories.remove("Choice")
        val initialState = balutGameManager.initializeGame()
            .copy(playerScores = mapOf(0 to allCategories))
        val updatedState = balutGameManager.scoreCategory(
            initialState, listOf(1,2,3,4,5), "Choice")
        assertTrue(updatedState.isGameOver)
        assertEquals(15, updatedState.playerScores[0]?.get("Choice"))
    }
    \end{lstlisting}
\end{itemize}

\section{Detected and Fixed Bugs}

During testing, several issues were identified and subsequently fixed.  Notable bugs included:

\begin{itemize}

    \item \textbf{Shake Detection Sensitivity}: Rapid shaking triggered multiple dice rolls, fixed with a debounce mechanism.
    \begin{lstlisting}[language=Kotlin, caption=Fix for Shake Detection Sensitivity, label=lst:shake_detection_fix_code]
    fun resumeShakeDetection() {
        shakeDetectionManager.setOnShakeListener {
            viewModelScope.launch {
                shakeFlow.emit(Unit)
            }
        }
        viewModelScope.launch {
            shakeFlow
                .debounce(300) // Added 300ms debounce
                .collect {
                    if (!isRolling.value && isRollAllowed.value) 
                        rollDice()
                }
        }
    } 
    \end{lstlisting}

    \item \textbf{Settings Navigation State Loss}: Navigation to the settings screen reset the game state; addressed by using launched effects for navigation to ensure game state retention.
    \begin{lstlisting}[language=Kotlin, caption=Fix for Settings Navigation State Loss, label=lst:setting_navigation_fix_code]
    LaunchedEffect(selectedBoard) {
        // Only reset game if board type changes
        if (selectedBoard != currentBoardType) {
            viewModel.setSelectedBoard(boardType)
            viewModel.resetGame()
        }
    } 
    \end{lstlisting}

    \item \textbf{Game State Initialization}: Game state not correctly initialized leading to incorrect player scores. Was fixed by ensuring the game state was reset correctly when starting the game.
    \begin{lstlisting}[language=Kotlin, caption=Fix for Game State Initialization, label=lst:game_state_reset]
    fun resetGame() = viewModelScope.launch {
        _showWinDialog.value = false
        _heldDice.value = emptySet()
        diceManager.resetGame()
        statisticsManager.startGameTiming()
    } 
    \end{lstlisting}
    \item \textbf{UI Responsiveness}: Some UI elements were unresponsive on certain devices. This was resolved with optimization in UI rendering to render responsive on diverse display dimensions.
   
    \item \textbf{Dice Recognition Accuracy}: Dice value misidentification; fixed by refining the image processing techniques with a min-max normalization of pixel brightness values to enhance pips from backgrounds.
\end{itemize}

\section{Results of Experiments}

Throughout the development of the application, several experiments were conducted to assess the performance and usability of the system. The following results were observed:

\begin{itemize}
   \item \textbf{Performance Metrics}: The application maintained a consistent frame rate of 60 FPS during gameplay, regardless of whether the player was engaged with a single opponent or multiple opponents. This demonstrates the application's smoothness and efficiency in rendering, which is crucial for delivering a responsive and enjoyable user experience.
   
   \item \textbf{User Feedback}: User testing was conducted with a diverse group of participants. The feedback gathered indicated high levels of satisfaction, with an average rating of 4.5 out of 5 for both the user interface and overall functionality. This suggests that the design is intuitive, easy to use, and meets the expectations of the target audience.
   
   \item \textbf{Code Quality and Maintainability}: During development, efforts were made to maintain a high standard of code quality. This was reflected in the following practices:
   \begin{itemize}
       \item \textbf{Clear Documentation}: Each method and class was thoroughly documented using consistent header comments, making it easier for future developers to understand and extend the codebase.
       \item \textbf{Modular Design}: The code was structured to follow principles of modularity, with components designed to be reusable and maintainable. This approach facilitates future improvements and reduces the risk of introducing bugs.
       \item \textbf{Efficient Code Practices}: The code was optimized for performance, and redundant logic was removed to ensure that the application runs efficiently, even as new features are added.
       \item \textbf{Consistent Coding Standards}: A consistent coding style was enforced throughout the project, making the codebase easier to read and reducing cognitive load when navigating different parts of the application.
   \end{itemize}
\end{itemize}
