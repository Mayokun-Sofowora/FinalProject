
\chapter{Conclusions}

This chapter summarizes the project's key findings and achievements, reflecting on the objectives outlined in the thesis, the challenges encountered during development, and potential paths for future enhancements.

The game project successfully met its primary objectives, resulting in a modern Android application that implements multiple classic and custom dice game variants. The application features three distinct game variants: Pig, Greed, and Balut each with unique rules and gameplay mechanics, which were designed to enhance user engagement and provide a comprehensive gaming experience. An adaptive AI opponent was also developed to challenge players, adjusting its difficulty based on their performance. This AI provides a more engaging experience, and encourages strategic thinking, making the game more dynamic. The application also boasts a user-friendly interface designed with a modern Material Design 3 UI, ensuring an intuitive user experience. The implementation of customizable themes and touch controls enhances accessibility and user satisfaction. The application follows MVVM and Clean Architecture principles, which promote maintainability and scalability.  Dependency injection using Hilt and reactive programming with Kotlin Coroutines and Flow were effectively utilized. Finally, the application was validated with a thorough testing strategy, including unit tests and integration tests, which ensured the reliability and stability of the application.

Throughout the development, several challenges were encountered. Implementing the various game rules and ensuring accurate scoring mechanisms proved complex, requiring extensive testing and debugging. Creating an adaptive AI that could effectively challenge players was also a significant hurdle, requiring much trial and error to balance the AI’s difficulty level. The design of a user-friendly interface that accommodates various screen sizes also required careful consideration and multiple iterations to achieve the desired outcome. Furthermore, implementing user authentication and data synchronization presented complexities. Setting up and managing user authentication with Firebase, navigating its documentation, and handling different authentication flows was challenging. Similarly, ensuring seamless data synchronization between Firebase and Android's DataStore required careful management of data consistency and conflict resolution. Finally, the implementation of image recognition was also difficult. The first attempts to train a custom model with TensorFlow Lite, was difficult and very time consuming, and in the end, Roboflow had to be adopted as a solution for its ease of use and effectiveness in handling image recognition tasks.

While the project has achieved significant milestones, several avenues for future development remain. Future updates could include additional game variants or modes, such as multiplayer options or online leaderboard. This would be useful to enhance competitiveness and social interaction among players. Further development could also focus on improving the AI's decision-making algorithms, and providing the option to select the difficulty of the AI. Integrating augmented reality (AR) elements could also provide a more immersive gaming experience, allowing players to interact with the game in new and innovative ways. Expanding the application to support other platforms, such as iOS or web-based versions, could also broaden the user base and increase accessibility. Finally, the implementation of a feedback mechanism within the app could help gather user insights, guide future enhancements, and ensure that the application continuously meets user expectations.